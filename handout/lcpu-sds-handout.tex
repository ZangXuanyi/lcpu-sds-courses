\documentclass[12pt]{report}
\usepackage[UTF8]{ctex}
\usepackage{geometry}
\usepackage{amssymb}
\usepackage{amsmath}
\usepackage{amsfonts}

%listings
\usepackage[most]{tcolorbox}
\usepackage{xcolor}
\usepackage{listings}

\newtcblisting{codebox}[1][]{
    listing only,
    sharp corners, % 直角
    colback=blue!5!white, % 背景色
    colframe=blue!50!black, % 边框色
    boxrule=1pt, % 边框粗细
    left=5pt,right=5pt,top=5pt,bottom=5pt, % 内边距
    #1
}

\title{LCPU 先导课讲义}
\author{2300013192}
\date{2025年4月}
\geometry{a4paper, margin=1in}

\begin{document}
\maketitle
\tableofcontents

\setcounter{chapter}{-1}

\chapter{引言} % Chapter Over

% 使用示例:
\begin{codebox}[title=你好,计算机!]
    #include <stdio.h>
    int main() {
        printf("Hello, World!\n");
    }
\end{codebox}

计算机基础科学教育是北京大学近些年一直努力推进的教育之一,体现在北京大学的所有学生都应修习《计算概论》这一门课程。然而,因为众所周知的原因,高中乃至初中对于计算机的教育水平参差不齐,因此同学们的基础也不尽相同,这导致部分基础较差的同学在学习《计算概论》时会遇到困难。

一方面为了帮助同学们修习《计算概论》,另一方面为了帮助同学们更好地了解计算机以方便同学们在之后的学习和科研,北京大学学生Linux俱乐部(LCPU)开设了《LCPU先导课》这一门课程。比起“CPU是怎么构成的”“软件是怎么工作的”这种理论知识,这门课更注重于“我们应该购买什么CPU”“我们应该怎么搭建一个软件开发平台”这类的实用知识。

本课程的目标是将同学们的计算机水平快速提升至能够接受大学计算机基础教育的水平。我们认为来上本课程的同学都已经具备了最基本的计算机操作能力,也就是说我们不会安排类似“怎样使用鼠标”这种内容。

希望本课程能够对同学们有所帮助。

\chapter{计算机基础知识} % Chapter Over

\section{计算机的鼻祖}

一般认为,图灵是赋予现代计算机灵魂的人。他在1936年提出了“图灵机”的概念,图灵机是一个抽象的计算模型,能够模拟任何计算机的计算过程。自此,问题被分为两类:能用图灵机解决的(或者称作“图灵可计算的”)和不能用图灵机解决的;遇到前者,我们就可以掏出图灵机计算。图灵机的提出为现代计算机科学奠定了基础。关于图灵机本身的内容已经超出了本节课的范围,我们在这里不做过多介绍。

然而,图灵机更多的是一种抽象概念,真正把计算机变成现实的是冯·诺依曼。冯·诺依曼在1945年提出了“冯·诺依曼架构”,也就是我们现在所说的计算机的基本结构。冯·诺依曼架构的核心思想是将程序和数据存储在同一块内存中,这样计算机就可以根据程序的指令来操作数据。冯·诺依曼架构是现代计算机的基础,目前几乎所有的计算机都遵循这一架构。

\section{现代计算机的组成}

我们现在的计算机,俗称“电脑”,是由硬件和软件两部分组成的。硬件就像人的四肢,而软件就像人的大脑。在实际使用中,硬件是我们可以直接看到和摸到的,而软件我们是无法直接接触的。

\subsection{计算机的硬件}

硬件,也可以叫做设备,可以简单地分为两类:输入输出设备和主机设备。以一台笔记本为例,直接露出在外面的几乎都是输入输出设备,例如显示器、键盘、触控板、USB接口等;而主机设备则是隐藏在笔记本内部的,例如CPU、内存、硬盘等。

\textbf{CPU}是计算机的最核心部件,它从存储设备读取指令和数据,并且执行这些指令。尽管现代处理器对代码和数据会有不同的处理,但是其本质上并没有严格的区分。代码由一条一条的指令组成,CPU 按照顺序一条一条执行从存储设备中读取的指令(至少从软件和程序员等使用者的视角看是这样),指令可以是修改 CPU 的状态,进行运算,或者是从其他硬件读取信息或者输出信息。如果希望进一步学习CPU如何运作等相关知识,可以参考著名的教材《CSAPP》,也可以修习《计算机系统导论》(ICS)这门课。

\textbf{内存}是计算机的临时存储器,它用于存储正在运行的程序和数据。它能够被CPU直接访问,因此速度较快。对于程序员而言,内存可以被抽象为一堆连续的存储单元,每个存储单元都有一个唯一的地址;执行程序时,程序的一部分或者全部被放进内存中,CPU就在内存中找寻需要的数据或者指令,如同在排列整齐的暑假上寻找需要的书籍。

现代计算机内存读写速度很快,但是已经跟不上CPU的速度,因此又引入了高速缓存来加速内存的读写速度。高速缓存是内存和CPU之间的一个小型存储器,它存储了最近使用的数据和指令,以便CPU可以更快地访问它们。在断电以后,内存中存储的数据会丢失,因此内存也被称为是易失性的存储器。

特别注意:上述文本中的“内存”指的是“随机存取存储器”(RAM)。这里的“随机”指的是“随机存取”,也就是可以在任意时刻访问任意地址的存储器,而不是“顺序存取”的存储器(例如磁带)。同时,“内存”这个词在部分语境下又不同的含义,例如在BIOS语境下的“内存”指的是“只读存储器”(ROM),在移动设备(手机)等语境下的“内存”指的是“闪存”,这实际上是外存。

\textbf{主板}连接了所有的硬件设备,也是核心部件之一。主板上有很多插槽和接口,用于连接CPU、内存、硬盘、显卡等设备。使得整个计算机形成一个电路系统。主板也能够为主芯片的工作提供各类条件支持(如供电降压,数据的传输,机械支撑,芯片工作顺序的控制等),提供固件进行硬件自我检查和系统初始化,保存各个硬件的配置等。

\textbf{外存}是现代计算机的主要存储设备,用于存储操作系统、应用程序和数据等内容。其读写速度往往比内存慢得多,但是它的存储容量更大且往往是非易失的(相对内存而言)。

现代计算机的主要外存设备是硬盘。硬盘可以分为机械硬盘(HDD)和固态硬盘(SSD)。机械硬盘使用磁头在旋转的磁盘上读取和写入数据,而固态硬盘使用闪存芯片来存储数据。固态硬盘的读写速度比机械硬盘快得多,现在价格也便宜得多,但是使用寿命较短,且因为电荷流失等问题无法接受长期不通电等情况,不适宜作为长期存档介质(个人使用寿命和HDD无明显差异,基本都能用到彻底换机)。

除硬盘外,还有其他外部存储设备。例如:
\begin{itemize}
    \item U盘:一种小型的闪存存储设备,通常通过USB接口连接到计算机上。本质是一个弱化的SSD,它可以临时地用于存储和传输数据。
    \item 光盘:一种使用激光读取和写入数据的存储介质。常见的光盘有CD、DVD和蓝光光盘。在大量存档的时候较为经济(大量购买比HDD便宜,可靠性比SSD高)。缺点是容易划伤和损坏,且信息密度低,读写速度慢。
    \item 磁带:一种使用磁性材料存储数据的介质,通常用于备份和存档。磁带的读写速度极为缓慢(和倒带速度成正比)且需要专门的设备来读写,设备价格昂贵,维护成本高。其优点是可靠性高,存储密度高。
    \item 软盘:一种老古董,使用磁性材料存储数据。软盘的存储容量较小,已经基本被淘汰。
\end{itemize}

提醒:\textbf{硬盘有价,数据无价。}请务必定期备份数据,尤其是重要数据。

\textbf{显卡}是计算机的图形处理器,它用于处理图形和视频数据。显卡可以加速图形渲染,提高游戏和视频播放的性能。显卡通常有自己的内存,用于存储图形数据。

对于现在AI时代而言,显卡因为其良好的并行特性使得其成为了深度学习的首选硬件。显卡的计算能力通常用“浮点运算每秒”(FLOPS)来衡量,通常情况下,显卡在机器学习等需要大量并行的简单计算工作上,表现远好于CPU。

\textbf{电源}是计算机运行不可或缺的一部分,它虽然不参与数据存储、指令执行等工作,但是能够向计算机不同的部件提供所需的电压和电流(对CPU而言工作电压仅1V,但是电流可达数百安培)。优质的电源能够向计算机供给稳定的电压和电流,避免计算机在运行过程中出现故障,提高计算机的寿命。

\textbf{输入输出设备}是用户与计算机交互的直接界面。以前这个交互是拔插电缆,后来转变为了打孔纸带,一直发展到现在的键盘、鼠标、显示器等丰富的交互设备。

\subsection{软件}

\textbf{操作系统}是计算机的核心软件,它负责管理计算机的硬件和软件资源。操作系统提供了一个用户界面,使用户可以与计算机进行交互。我们可以认为操作系统是连接现代软件和硬件的桥梁。目前,常见的操作系统有Windows、macOS、Linux等。

Windows是目前占有市场份额最大的操作系统。它由微软公司开发,广泛应用于个人计算机。Windows以其易用性和兼容性而闻名,广泛支持各种软件和硬件设备,但是缺点是不适宜用作开发工具。

macOS是苹果公司开发的操作系统,专门用于苹果的计算机产品。macOS以其优雅的界面和强大的功能而闻名。

Linux是一个开源的操作系统,它是一个类Unix操作系统。Unix因为太大了而不适宜在个人计算机上使用,因此Linus等人开发了Linux,但因为学习曲线陡峭,至今未能广泛应用于个人计算机上,服务器和嵌入式系统使用居多。Linux的开源特性使得它可以被自由修改和分发,因此有很多不同的Linux发行版,例如Ubuntu、Debian、Arch等。

对于计算机新手,我们推荐使用Windows和macOS系统作为操作系统,这是因为它们提供了友好的用户界面和丰富的软件支持,适合初学者使用。对于希望深入学习计算机的初学同学,我们推荐使用Linux系统的发行版Ubuntu,因为它具有和Windows与macOS类似的图形界面,并具有良好的社区支持和丰富的学习资源。对于希望进阶的同学,我们推荐使用Arch Linux,它是一个轻量级的Linux发行版,具有高度的可定制性和灵活性。

\textbf{驱动程序}是操作系统和硬件之间的桥梁,它负责将操作系统的指令转换为硬件可以理解的语言。驱动程序通常由硬件制造商提供,并且在操作系统安装时自动安装。驱动程序的作用是使操作系统能够正确地识别和使用硬件设备。

驱动程序通常是特定于硬件的,因此不同的硬件设备需要不同的驱动程序。操作系统通常会自动检测硬件设备并安装相应的驱动程序,但是有时候需要手动安装驱动程序。我们可以到软件官网上下载最新的驱动程序,或者使用操作系统自带的驱动程序更新工具来更新驱动程序。不推荐使用“驱动精灵”等第三方驱动程序更新工具,因为它们可能会安装不必要的驱动程序,甚至可能会导致系统不稳定。

\textbf{常规软件}是我们具体用于实现某一功能的工具。这类软件多得很了,我们常用的通讯软件QQ、微信等,浏览网页的Chrome、Microsoft Edge等,都是常规软件。
我们下载软件的主要渠道有两种:通过官方渠道下载、通过包管理器(Winget,Homebrew,apt等)下载。这两种渠道一般认为是最安全且问题最少的。

以安装QQ为例,在Windows上,我们需要在腾讯官网上找到QQ的下载页面,然后下载并安装之。在有包管理器的系统(如Linux、macOS)上,我们可以通过包管理器下载,例如brew qq(不保证该命令能够执行。你可能需要基于不同系统使用其他的一些命令。)。我们并不推荐在非官方渠道下载软件,这些非官方渠道往往以某某软件站、某某下载站、某某应用商店(Microsoft Store这类系统自带的除外)等形式出现。通过上述方式下载的软件可能导致使用盗版、附带流氓插件甚至木马、病毒等问题,或者遇到一些其他各种问题。

\section{计算机间的通讯}

除了使用光盘、U盘等存储介质传递信息以外,计算机还可以通过组网来进行计算机间的通讯。我们最常见的网络是“因特网”,它最初是由美国军方开发的一个网络,后来逐渐发展成为全球最大的计算机网络。因特网是一个分布式的网络,由无数个互相连接的计算机组成。因特网的核心是一组名为TCP/IP协议的网络协议,它定义了计算机之间如何进行数据传输。

上述协议具体是怎么工作的已经远远超出了本节课的范围,我们在这里不做过多介绍,希望进一步学习这些内容的同学可以参考其他课程,例如《计算机网络》这门课。

\chapter{计算机初步实践}

\section{计算机的选购}

个人使用的微型计算机可以简单地分为台式机和笔记本两类。显然,台式机不便于移动,但是往往性价比更好,且性能更加强大;而笔记本则便于携带,但是溢价较为严重。

\begin{codebox}[title=选购笔记本的建议]
    NotImplementedError("TODO: 选购笔记本的建议")
\end{codebox}

\section{连接北京大学校园网}

\section{北京大学正版软件}

\section{操作系统的简单维护}

\subsection{及时修补漏洞}

\subsection{定期备份数据}

\subsection{定期清理系统垃圾}

\subsection{碎片整理与磁盘剪裁}

\section{计算机的简单进阶使用}

\subsection{更好的终端}

\subsection{包管理器}

\subsection{版本控制} % Subsection Over

试想以下环境:我们正在写一项作业,开发工作已经基本完成,试运行也能够得到90分。此时我们希望进一步精进代码,使得分数达到95分以上;但是经过一通修改以后,发现程序再也运行不起来了。这时候距离ddl只有1小时,我们决定摆烂,提交能够得到90分的代码。然后我们根据记忆改回原来的代码的时候,发现我们再也想不起来旧代码是怎么写的了!这无疑是令人极为懊恼的。

为了避免以上问题,我们引入了版本控制系统。目前最常用的版本控制系统是Git。

有些同学可能会问Git和GitHub有什么关系,在此我的回答是他俩的关系和Porn和PornHub的关系差不多。

\subsubsection{Git的工作原理}

Git有三个目录共同完成版本控制:工作区、暂存区、版本库。工作区是项目目录,暂存区是一个隐藏的文件夹.git,版本库是一个隐藏的文件夹.git/objects。工作区是我们平时使用的目录,暂存区是Git用来存储修改的地方,版本库是Git用来存储所有版本信息的地方。版本库有一个指针,指向当前版本的某一节点(一般指向最新的节点)。每个节点都有一个唯一的哈希值,用来标识该节点。每个节点包含了该版本的所有文件和目录的信息,以及指向上一个版本的指针。Git使用哈希值来标识每个版本,这样可以保证每个版本都是唯一的。

这样讲解很难以理解,我们不妨举例说明:现在,Git中有一个版本为X的节点,包括文件A和文件B两个文件。这些文件存储在版本库中。此时,工作区为空,暂存区为空,指针指向X。我现在希望对它们进行修改,这个修改遵循以下过程:

\begin{enumerate}
    \item 我拿出了这些文件,并且对文件A进行修改。此时,工作区有AB两个文件,但是暂存区依然是空的。我们的任何修改都不会被暂存区记录,Git也不会知道我对这些文件进行了修改。
    \item 我觉得修改差不多了,现在把A放进暂存区。现在Git知道我对A进行了一些修改了。
    \item 我又对B进行了类似的修改,此时B也进暂存区了。
    \item 我觉得修改差不多了。我认为我应该永久保存目前的状态,于是就把暂存区提交到版本库。此时版本库多了一个Y节点,指针也指向Y节点,有修改过的AB两个文件。此时,暂存区又清空了,而工作区和版本库的Y版本一致。
\end{enumerate}

\subsubsection{下载Git}

一个最简单的方式是使用Winget包管理器:

\begin{codebox}[title=Windows下安装Git]
    winget install git
\end{codebox}

或者你也可以从官方网站上下载并安装之。同样,安装的时候一定要勾选“添加到PATH”这一选项,否则你在命令行中无法使用Git。

\subsubsection{Git信息设置}

安装并使用Git的第一步是先编辑本地的一些提交信息。Git的提交需要一个用户名和一个邮箱,来对应每次提交的作者。我们可以使用以下命令来设置这些信息:

\begin{codebox}[title=设置Git用户名和邮箱]
    git config --global user.name "Your Name"
    git config --global user.email "email@example.com"
\end{codebox}

这样即可设置全局用户名和邮箱。如希望给某个特定仓库设置特定的用户名和邮箱,你需要在该仓库下重新执行上述命令,但是不写--global命令。

\subsubsection{Git版本控制:提交}

要具体地在某一目录下进行版本控制,我们需要在命令行中进入到我们希望使用Git的目录下。然后我们可以使用以下命令来初始化一个Git仓库:

\begin{codebox}[title=初始化Git仓库]
    git init
\end{codebox}

如果你在视窗中开启了“显示隐藏文件”这类功能,你就会发现一个隐藏的文件夹.git出现在了你当前的目录下。这个文件夹就是Git用来存储版本信息的地方。

然后你可以使用以下命令来添加文件到Git仓库中(这个命令的实际意义是把文件添加到暂存区);

\begin{codebox}[title=添加文件到Git仓库]
    git add <filename>
\end{codebox}

如果我们忘记了当前状态下有哪些文件被修改了,我们可以使用以下命令来查看当前状态:
\begin{codebox}[title=查看当前状态]
    git status
\end{codebox}

如果你觉得修改差不多了,保存文件以后,你可以使用以下命令来提交文件到Git仓库中(这个命令的实际意义是把暂存区的文件提交到版本库中):

\begin{codebox}[title=提交文件到Git仓库]
    git commit -m "commit message"
\end{codebox}

上述内容中,-m后面是提交信息。提交信息是对本次提交的简要描述。我们建议每次提交都写上简要的提交信息,这样可以帮助我们更好地理解代码的修改历史。

\subsubsection{Git版本控制:回退}

如果出现了先前我们说的不小心写坏了的情况,这时候就可以进行版本回退了。我们可以使用以下命令来查看当前的版本信息:

\begin{codebox}[title=查看版本信息]
    git log # 例如版本库是a-b-c-d-e-f-g
\end{codebox}

找到你希望回退到的版本的哈希值(前几位即可),然后使用以下命令来回退到该版本(这个命令会把指针回退到指定的版本,丢弃之后的所有内容,然后丢弃暂存区和工作区的所有东西):

\begin{codebox}[title=回退到某个版本]
    git reset --hard <commit_hash> # 请谨慎使用这一命令!该命令不会保留当前的修改!
\end{codebox}

如果你希望回退到某个版本,但是不想丢失当前的修改,你可以使用以下命令来回退到该版本(这个命令会把版本库后面的东西全部丢弃,清空暂存区,但是保留当前工作区):
\begin{codebox}[title=回退到某个版本]
    git reset --mixed <commit_hash> # 我们更加推荐这个回退方式,--mixed可以省略,或者用--soft替代。
    用--soft替代时,不会清空暂存区。
\end{codebox}

如果你只是想看看老版本的内容而不回退,你可以使用以下命令来查看某个版本的内容(这个命令不会改变版本库和指针,但是会清空暂存区和工作区,并进入分离头模式)

\begin{codebox}[title=查看某个版本的内容]
    git checkout <commit_hash> # 我不回退,我就看看
    git checkout <branch_name> # 回到最新版本
\end{codebox}

所谓分离头模式,指的是在git checkout <commit\_hash> 的时候(假如checkout e),并且你没有回到最新版本就又进行了一次commit,节点是h,那么就会新产生一个树e-h。

\subsubsection{Git版本控制:排除}

有时候我们版本跟踪的时候不需要跟踪一些文件,例如具有敏感信息的文件(如密码),或者构建文件等。此时,我们可以创建一个文件 .gitignore 来阻止跟踪。例如,在Linux下,构建文件往往是*.o。那么我们可以在上述文件中加入 *.o ,之后git就会忽略这些文件。

关于Git版本控制的一些更加进阶的知识(例如分支管理等内容),欢迎查阅更多资料。

\section{网安相关知识} % Section Over

\subsection{网络的风险:钓鱼、木马、蠕虫和病毒}

虽然互联网的出现给我们带来了便利,但也带来了很多风险。部分人心术不正,使用互联网进行诈骗、盗窃、敲诈勒索等违法犯罪活动;而他们使用的主要手段是不定向的网络攻击,例如钓鱼、木马和病毒等。

钓鱼指的是使用伪造的网页、邮件等方式,诱骗用户输入个人信息,例如用户名、密码、银行卡号等。其目的通常是获取用户的私人信息,以方便对其进行后续的诈骗、勒索等活动。

木马这个词来源于神话中的“特洛伊木马”,原指在一只巨大的木马中藏匿士兵,诱骗敌人打开城门进而发动攻击。现在的木马指一种恶意软件,它伪装成合法的软件或者捆绑在合法软件中,诱骗用户安装。一旦安装,木马就可以在用户不知情的情况下,窃取用户的个人信息、打开端口等。例如,一种最古老且经典的木马是FTP木马,它会在用户的计算机上打开一个FTP端口,允许攻击者远程访问用户的计算机。

蠕虫指的是一种自我复制的恶意软件,它可以在计算机之间传播。蠕虫通常利用计算机系统的漏洞进行传播,一旦感染了一个计算机,就会自动复制自己并传播到其他计算机。蠕虫通常会消耗计算机的资源,导致计算机变得缓慢或者崩溃。一个很经典的蠕虫是“小邮差”,它通过发送带毒邮件进行传播,会占满计算机的网络带宽;一旦感染了计算机,就会自动发送带毒邮件给其他计算机。

病毒指的是一种恶意软件,它可以在计算机之间传播。病毒通常依附在合法的软件中进行传播,一旦感染了一个计算机,就会自动复制自己并传播到其他计算机。病毒通常会破坏计算机的文件、数据等,导致计算机无法正常工作。一个很经典的病毒是“CIH”,它会在每年的4月26日感染计算机,并破坏计算机上的所有文件。它与蠕虫的区别是,病毒不能自己执行,只能依附于其他软件执行;而蠕虫可以自己执行。

被上述恶意软件感染后,计算机会变得不稳定、产生额外的资源开销,甚至导致计算机崩溃、数据破坏,造成经济或其他损失。君子爱财取之有道,我们应该遵纪守法,不要为了炫耀技术或者获取经济利益而制作这些恶意软件。

\subsection{从根源断绝问题}

为了防止计算机受到感染和破坏,最简单的方式是从根源上解决问题。我们在日常使用网络的时候,应该遵循以下原则:

\textbf{不浏览不安全网页}:在浏览网页的时候,若不能确定安全性,则尽量避免浏览不明链接、下载不明文件;如果确实需要下载软件,应该到官方网站上下载。

\textbf{识别伪造网站}:当一个网站需要你输入个人信息时,应该仔细检查该网站的安全性;钓鱼网站通常会伪装成合法网站,例如使用HTTPS协议、与合法网站相似的域名等。我们可以通过查看浏览器地址栏中的锁图标、检查网站的证书、观察域名是否正确等方式来判断网站的安全性。北京大学计算中心每年都会主动制作钓鱼网站,测试本校教职工和学生抗钓鱼的能力。虽然被计算中心骗了是一件不太光彩的事情,也总比被其他人骗了好,至少不会损失钱财。

\textbf{保持系统和软件的更新}:系统和软件的更新有一大部分是安全更新。定期更新软件可以阻止恶意软件利用漏洞进行攻击。

\textbf{保持杀毒软件的自动检测功能开启}:这类检测功能可以帮助我们初步检测计算机上的恶意软件。虽然有时候存在令人诟病的误报,但是它们仍然是我们保护计算机的一道重要防线。

\textbf{使用密钥代替密码}:密钥是一种更安全的身份验证方式,它可以防止密码被窃取。除了常用的公钥-私钥对,密钥还可以是USB设备、手机等。使用密钥可以防止密码被窃取和破解。目前一些技术网站已经支持使用密钥登录,例如GitHub、Google等。而我们在登录远程服务器的时候,也建议使用密钥而非密码登录。

\textbf{使用强密码并定期更换}:如果不得不使用密码登录,建议使用复杂的密码,并定期更换密码。复杂的密码应该包含字母、数字和特殊字符,并且长度至少为8位。我们也可以使用密码管理器(例如BitWarden)来生成和管理复杂的密码。

\textbf{定期备份数据}:定期备份数据可以防止数据丢失和损坏。数据备份有一个321原则:3份数据,2种介质,1个异地备份。也就是说,我们应该至少有3份数据备份,其中2份存储在不同的介质上(例如移动硬盘和云存储),1份存储在异地(例如云存储)。这样,即使我们的数据损坏了,也可以迅速恢复来减少损失。

\textbf{善用沙箱}:沙箱是一种虚拟化技术,可以将应用程序隔离在一个独立的环境中运行。这样可以一定程度上防止恶意软件对计算机造成损害。我们可以使用虚拟机、Docker等工具来创建沙箱环境,对于不能确定安全性的程序可以在此类环境中运行。

\subsection{亡羊补牢,为时未晚}

如果发现自己被钓鱼,应该紧急冻结相关账户并迅速更改密码,防止进一步的经济损失。在接下来的一个月到数个月中务必慎之又慎,你的信息可能已经被泄露,招致电信诈骗的概率显著增高。

如果发现自己的计算机感染了木马、病毒、蠕虫等,此事比被钓鱼更加严重。你应该依次执行以下内容:

\begin{itemize}
    \item \textbf{立即结束进程}:如果你能定位具体是哪一个软件正在搞破坏,可以使用任务管理器结束该进程。不过大多数情况下我们无法确定具体是哪一个进程出问题,此时忽略这一步即可。
    \item \textbf{立即断网}:这是一种负责任的行为,可以防止病毒进一步扩散。仅在软件层面上切断网络并不足够,如果你的计算机使用有线网络应该拔掉网线,以防止恶意进程重新连接网络。
    \item \textbf{立即查杀}:你可以运行杀毒软件的查杀功能,如是旧种类的病毒,杀毒软件应对起来不会太难。
    \item \textbf{立即上报}:如果杀毒软件查杀失败,应该立即上报相关部门。这可能意味着新型病毒的出现,上报有关部门有利于他们做出迅速反应,可以减少损失。
    \item \textbf{立即重装系统}:彻底重新格式化硬盘并重装系统可以彻底地清除残留的病毒文件。这是没有办法的办法,但却是最有效的。
\end{itemize}

\chapter{搜索和信息获取} % Chapter Over

在大学,上课和课本固然是一种重要的信息获取方式。但是课程和课本本身由于是静态的,往往无法及时更新最新的信息;然而对计算机科学等发展迅速、信息爆炸、对技术要求较高的科目,仅凭借课本等静态资源显然是远远不够的。因此,我们需要借助其他的方式来获取信息。

\section{搜索}

\subsection{搜索引擎的选择}

国内最常见的搜索引擎是百度。但是当我们在百度搜索相关内容时,第一页往往会被大量的广告占据。这显然并不是我们想要的结果。其他的国内搜索引擎都或多或少有相关的问题,因此并不好用。

由于现在大多数新购整机都预装了Windows正版系统,因此基本上都自带一个内置浏览器 Microsoft Edge。Edge 的默认搜索引擎是必应(Bing),在搜索的时候我们可以在页面顶端发现“国内版”和“国际版”的选项。在使用国内版搜索时,仍然会出现少量的广告和不相关的内容,但是相对百度而言,必应的搜索结果要好得多。在使用国际版搜索的时候,必应的搜索结果会更好,但是由于网络问题,可能会出现无法访问的情况。

细心的同学可能会发现,我们从国内网络访问Bing,无论是国内版还是国际版,网址都是cn.bing.com。而真正的Bing的网址是www.bing.com。有条件能够访问这一网址的同学可以使用这个Bing。而Google作为全球最大的国际搜索引擎,搜索结果通常比必应还要准确、直接且全面。

因此,不使用特殊方式上网的情况下,如果我们要搜索的信息\textbf{非中文社区独有},我们更推荐使用必应的国际版搜索引擎。在该课程中不将涉及任何特殊上网方式的教学。

\subsection{搜索技巧}

有时候我们搜索的时候无法搜索到想要的信息。这时候我们需要使用一些技巧。

\textbf{关键词搜索}是最常见的搜索技巧之一。我们使用完整句子进行搜索的时候,搜索引擎会利用语言模型将其拆分成多个关键词进行搜索,而语言模型总会导致一定的偏差。因此,我们可以一步到位,使用关键词进行搜索。例如,我们如果想要搜索“我怎样改善睡眠质量”,可以把它拆分成“改善睡眠 方法”关键词进行搜索;如果需要进一步约束(例如我希望方法快速起效),可以搜索“改善睡眠 方法 快速”。

\textbf{使用英文}是另一个常见的搜索技巧。中文互联网的一大特点是信息向应用内部收缩,形成无法被搜索引擎检索到的“深网”,导致中文开放互联网的信息量小于英文开放互联网的信息量。使用英文搜索的另一个原因是英语依然是世界上最通用的语言,尤其在技术、科学等领域,大部分的文献、资料、教程、说明等都是用英文写的;相关领域的研究材料往往也先以英文发表。因此我们在搜索的时候,使用英文搜索往往能够得到更好的结果。

即便英文水平一般的同学也不必担心。我们可以使用翻译软件(例如微软翻译、有道翻译等)将中文翻译成英文,然后再进行搜索。

\textbf{使用高级搜索选项}也是一种搜索技巧,最常见的高级搜索选项有:
\begin{itemize}
    \item 使用引号将关键词括起来,这样搜索引擎就会强制将其视为一个整体进行搜索,而不是将其拆分成多个关键词。依然以改善睡眠为例,可以使用“如何改善睡眠质量”进行搜索;
    \item 使用减号将不需要的关键词排除在外。例如,我们想要改善睡眠,但是不想看到关于药物的信息,可以使用“改善睡眠 方法 快速 -药物”进行搜索,这样搜索引擎就会把含有药物的信息排除在外;
    \item 使用“site:”限制搜索范围。例如,我们如果想要搜索“如何改善睡眠质量”,但是只想看到来自知乎的信息,可以使用“改善睡眠 方法 快速 site:zhihu.com”进行搜索。
\end{itemize}

\textbf{判断信息的可靠性}虽然不属于搜索技巧,但是却是一个非常重要的技能。我们在搜索到信息的时候,往往需要判断其可靠性。我们可以从以下几个方面来判断信息的可靠性:
\begin{itemize}
    \item 来源:信息的来源是否可靠?是否来自权威机构、专家或者知名网站?
    \item 时间:信息是否及时?是否过时?
    \item 评价:其他人对该信息的评价如何?是否有很多人认可?
    \item 完整性:信息是否完整?是否有遗漏?
    \item 可验证性:信息是否可以被验证?是否有相关的证据?
\end{itemize}

\section{信息平台}

除了使用搜索引擎在信息平台上搜索以外,我们还可以直接在著名的信息平台上面寻找相关信息。

\subsection{官方文档、官方Wiki、官方论坛}

如果我们希望获取某软件等的信息,最好的地方往往是其官方文档;对于类似于Arch Linux这种纯由社区维护的项目,其官方Wiki与论坛也是获取信息的最佳选择之一。官方文档虽然可能存在晦涩、难懂、省略等问题,但是往往依然是最权威、最全面的文档,这将会是你学习一门新技术的最佳选择。

如果你在请求问题的时候,遇到了诸如“RTFM”(Read The F**king Manual)的回应,这说明回答者认为你需要搜索官方文档和使用手册。当然在这种情况下,\textbf{他大概率是对的,你应该去读一读。}同样道理的还有STFW(Search The F**king Web)和RTFSC(Read The F**king Source Code)。而往往通过这种方式搜索信息,你能够学到的内容比直接告诉你答案要多得多。

\subsection{Stack Overflow}

堆栈溢出(Stack Overflow)是一个程序员问答网站,专门用于解决编程和技术问题。它是一个社区驱动的网站,用户可以在上面提问和回答问题。堆栈溢出有一个强大的搜索功能,可以帮助用户快速找到相关的问题和答案。从我的个人使用体验而言,这东西有点像百度贴吧和知乎的结合体,且专业性比两者都要强得多。

\subsection{GitHub}

GitHub是一个代码托管平台,用户可以在上面存储和分享代码。GitHub上有很多开源项目,用户可以在上面找到相关的代码和文档。GitHub还提供了一个强大的搜索功能,可以帮助用户快速找到相关的项目和代码。同时,GitHub也是一个非常重要的开源社区,当你对某个项目有疑问或者发现Bug的时候,你可以对该项目提出Issue,只要项目没“死”,总会有人告诉你答案;当你想要为某一项目做出贡献的时候,你可以Fork该项目,然后提交Pull Request。

\subsection{Wikipedia}

维基百科是一个自由的百科全书,用户可以在上面找到各种各样的信息。维基百科是一个社区驱动的网站,用户可以在上面编辑和修改条目。维基百科的内容是由志愿者编写和维护的,因此它的准确性和可靠性可能较低,不过它仍然是一个非常有用的信息来源。维基百科的搜索功能也很强大,可以帮助用户快速找到相关的条目。

\subsection{其他著名博客和教程}

\textbf{W3Schools}提供了许多关于开发的教程和示例,适合初学者使用。它的内容覆盖了HTML、CSS、JavaScript、SQL等多个领域。国内也有类似的网站,例如菜鸟教程、W3School等,只是内容丰富程度上较为逊色。

\textbf{OI Wiki、CTF Wiki、HPC Wiki}是一些关于算法、数据结构、编程竞赛等方面的Wiki,适合对这些领域感兴趣的同学使用。它们的内容覆盖了算法、数据结构、编程竞赛等多个领域。这些Wiki则是由在相关领域耕耘多年的选手前辈们维护的,内容质量较高。

\textbf{CS自学指南}是由信科的一位学长发起、旨在帮助计算机专业的同学自学计算机科学的一个项目。它的内容覆盖了计算机科学的各个领域,包括计算机网络、操作系统、编译原理等。它的内容质量较高,适合对计算机科学感兴趣且希望自学的同学使用。

\subsection{国内优质平台}

我们一般认为国内能够算上优质平台的有:博客园、哔哩哔哩、知乎、简书。这些平台普遍是免费的,你可以找到许多关于技术、编程、科学等方面的文章和视频。它们的内容质量参差不齐,也不乏卖课的(例如我曾经在B站看到过“预测2025年将会淘汰的编程语言:C/C++、Java、C\#、Golang、Python”等视频,当然这显然是胡扯),但是它们仍然是一个非常有用的信息来源。我们在接受信息的时候,仍然需要判断其可靠性。

特别说明:CSDN上虽然也有不少信息,但是该平台质量较低,商业化程度较高。这导致在该平台寻找信息的时候,我们必须在海量的AI水文、抄袭博客、低质付费文字、商业广告等无用信息中找到夹缝中的少数高质量文章,这是一件极为痛苦的事情。虽然在少数情况下我们最终能够找到一些有用的信息,但是\textbf{高质量的平台能节约鉴别信息的精力}。

\section{大语言模型}

现在,LLM已经广泛地投入了使用,无论是ChatGPT、Claude、Gemini等国外著名LLM,还是国内的DeepSeek、Kimi、QWQ等国内LLM,都已经投入了广泛应用。LLM使得我们获取信息的方法变得更加简单高效,我们可以把它们当作一个搜索引擎来使用。

与使用搜索引擎不同,我们在使用LLM获取知识的时候,需要遵从以下几点原则。

\begin{itemize}
    \item 具体性:使用LLM的时候提问应该极为具体,避免使用模糊、省略的语言或者关键字。例如我们如果想要获取改善睡眠质量的信息,应该使用“如何改善睡眠质量”而不是“改善睡眠”等关键字组合。
    \item 明确性:在使用LLM的时候,我们的Prompt应该明确无歧义。这在LLM上面有一个专门的课题叫做WSD(消歧)。例如一个著名的笑话“Have a friend for dinner”,我们应该明确地解释成“treat your friend to dinner”或者“Eat your friend”,而不是让LLM去猜。与之类似的是我们可以在Prompt中规定其输出格式,例如提供一个示例,这对获得期望的输出非常有效。
    \item 简单化:目前的AI依然缺乏处理复杂问题的能力。当我们提出一个复杂的问题时,LLM往往会混乱,进而得出错误答案。这时,我们可以采用分治思想,把一个大问题分成多个小问题,然后让LLM分别解决这些小问题,然后合并答案。
\end{itemize}

使用LLM其实并不容易,以至于现在有一个学科,叫做“提示词工程”,专门研究如何在不改进模型性能的条件下尽可能地优化其输出。有兴趣的同学可以自行查找相关资料。

\section{勇敢地提出问题}

当上述方法全部失败的时候,我们还有最后一个方法:可以抱大佬大腿,或者说向有经验的前辈提问和讨教。

除了抱身边大佬大腿以外,一个最传统的方式是,你可以在上述提到的平台或者其他技术社群上提问相关内容。你可以得到来自不同人的回答,这样你就有概率能够得到更多的帮助。当然,收集到的信息也相对良莠不齐,信息的价值需要自行甄别。同时,你的贴子和问答也会被其他人看到,一定程度上也可造福后人。例如,你可以在Stack Overflow上提出相关技术问题。

另一个方法是在GitHub上发布相关的Issue,这样项目的维护者就会看到你的问题,并提出相关的解答;有时候也有可能是项目本身的问题。这也能够帮助到以后的用户。

在提问的时候,应该遵照以下的原则:

\begin{itemize}
    \item 礼貌与尊重:没有人有义务解答你的问题,解决问题也许会耗费不少的时间和精力,大多数人解答问题往往只是出于本能的善意。礼貌的表达不仅能促使他人更愿意帮助你,还能建立良好的沟通氛围。当下互联网环境下,其实这一点的重要性远超想象。
    \item 增加有用信息:缺乏相关信息会让帮助你的人有心无力。程序崩溃有许多可能情况,不同的情况往往对应着不同的解决方案。如果能够在问题描述中增添足够的有用信息(例如列出错误代码),就会为解决问题增添巨大的可能性。
    \item 减少无用信息:部分人在提问的时候总会无意识地强调与问题无关的东西。这种内容往往会显著地降低信息密度,招致人的反感与厌恶。一个更常见的例子是在社群中发送大段语音而不是文字。
    \item 明确化你的描述:有时,我们的描述会出现歧义或者不明确的现象,例如“直面天命”这个短语对于没有关注或者没有游玩过《黑神话·悟空》的人而言容易导致迷惑。在这种情况下,使用更为具体的“游玩《黑神话·悟空》”等称谓更加合适。
    \item 列出你失败的尝试:这不仅表现出你为了自己解决自己遇到的问题所付出的努力,也能够显著地减少重复劳动与受到类似STFW等回复。
\end{itemize}

一个较好的提问例子是:

“我的电脑突然蓝屏了,我的蓝屏时候遇到的代码是 XXXXXXXX,是在游玩《黑神话·悟空》的时候突然蓝屏的。我上网搜索了代码相关的错误信息,尝试了网上可能有用的 A 方法和 B 方法,但都没有奏效。能麻烦你帮我看看吗?拜托了,非常感谢!”

\chapter{编程基础} % Chapter Over

以上都是计算机相关的基础知识,接下来我们将进入编程的世界。

无论我们学的是计算概论A还是BCD,都离不开编程。编程是计算机科学的核心技能之一,它使我们能够与计算机进行基本自由的交互,并利用计算机的强大计算能力来解决问题。编程不仅仅是一种技能,更是一种思维方式,它也能帮助我们更好地理解和解决问题。

\section{认识编程语言}

从现代意义上来说,最古老的编程语言当属“机器语言”。对于所有的机器而言,机器语言是最底层的语言,它是计算机能够直接理解和执行的语言。机器语言由二进制代码组成,每条指令都是一个二进制数。由于机器语言的复杂性和难以理解性,现代程序员通常不使用机器语言进行编程。

随着计算机科学的发展,渐渐地出现了“汇编语言”。汇编语言是一种低级语言,它使用助记符来表示机器语言的指令。汇编语言比机器语言更易于理解和使用,但是仍然需要对计算机的硬件架构有一定的了解。例如,我们需要针对x86架构写一个加法运算,我们需要以下的一大段代码:

\begin{codebox}[title=汇编语言加法运算,Windows,MASM语法]   
    .386
    .model flat, stdcall
    option casemap :none
    
    include \masm32\include\kernel32.inc
    include \masm32\include\msvcrt.inc
    includelib \masm32\lib\kernel32.lib
    includelib \masm32\lib\msvcrt.lib
    
    .data
        num1    dd 5          ; 第一个整数
        num2    dd 7          ; 第二个整数
        result  dd 0          ; 存储结果
        fmt     db "%d + %d = %d", 0
    
    .code
    start:
        ; 执行加法
        mov eax, num1         ; 将num1加载到eax寄存器
        add eax, num2         ; 将num2加到eax
        mov result, eax       ; 存储结果
        
        ; 打印结果
        push result           ; 第三个参数
        push num2             ; 第二个参数
        push num1             ; 第一个参数
        push offset fmt       ; 格式字符串
        call crt_printf       ; 调用C运行时库的printf
        
        ; 退出程序
        push 0
        call ExitProcess
    end start
\end{codebox}

上述代码并不简单,因此随着计算机科学的发展,出现了“高级语言”。高级语言是一种更接近人类自然语言的编程语言,它使用更易于理解的语法和结构。最先出现的一类高级语言是编译型的语言,代码文件先经过编译器编译成机器语言,然后再给计算机执行。这一过程是静态的,不便于调试,于是又出现了解释型语言,代码文件在运行的时候由解释器逐行解释执行。

最著名的编译型语言当属C系语言,它是由Dennis Ritchie在1972年开发的。C语言是一种通用的编程语言,它被广泛应用于系统编程、嵌入式系统、操作系统等领域。C语言的语法简单易懂,功能强大,因此成为了许多其他编程语言的基础。后来基于此又出现了C++,增加了面向对象的特性。

\begin{codebox}[title=编译型语言加法运算,C++语法]   
    #include <iostream>
    
    int main() {
        int num1 = 5; // 第一个整数
        int num2 = 7; // 第二个整数
        int result = num1 + num2; // 执行加法
        
        // 打印结果
        std::cout << num1 << " + " << num2 << " = " << result << std::endl;
        
        return 0; // 退出程序
    }
\end{codebox}

而最著名的解释型语言当属Python,它是由Guido van Rossum在1991年开发的。Python是一种高级编程语言,它具有简单易读的语法和强大的功能。Python被广泛应用于数据分析、人工智能、Web开发等领域。Python的语法简单易懂,功能强大,因此成为了许多初学者学习编程的首选语言。

\begin{codebox}[title=解释型语言加法运算,Python语法]   
    num1 = 5 # 第一个整数
    num2 = 7 # 第二个整数
    result = num1 + num2 # 执行加法
    
    # 打印结果
    print(f"{num1} + {num2} = {result}")
\end{codebox}

不过归根结底,编程语言只是工具。初学者在学习编程的时候,更应该关注的是编程的思想和方法,而不是具体的编程语言。每一门语言都有自己的长处和缺点,在实际使用的时候应该具体情况具体应对。

我们在计算概论课程上学习的两个主要语言是C++(不包含面向对象内容)和Python。接下来我不会讲述编程有关的内容(这是计算概论课程的内容),而重点放在“怎么能让你舒服地写程序”上:这就需要我们配置一个良好的编程环境。

\section{配置编程环境}

\subsection{C++编程环境}

刚刚提到,C++是一种编译型语言,因此我们需要一个编译器来进行相关操作。一般情况下,现在有三种通行的编译器:GCC、Clang和MSVC。一般情况下,GCC是跨平台的编译器,MSVC是Windows平台的编译器,而Clang在macOS上用得比较多。我们在这里推荐使用GCC编译器,因为它是一个开源的编译器,支持多种操作系统和平台;学校相关课程的自动测评用的也是这东西。

\subsubsection{Linux}

对于Linux,只需要在终端中输入以下命令即可一步到位:
\begin{codebox}[title=Linux安装GCC编译器]   
    sudo pacman -S gcc # Arch Linux,其他发行版请自行查找
\end{codebox}

然后如果我写好了一个代码文件hello-world.c,只需要在终端中输入以下命令即可编译:
\begin{codebox}[title=Linux编译*.c文件]
    gcc hello-world.c -o hello-world.o # 编译
    ./hello-world.o # 运行    
\end{codebox}

我们认为能够使用Linux开发的同学已经具有相当充足的计算机基础知识,因此不再赘述Linux的文本编辑器相关问题。

\subsubsection{Windows}

对于Windows,情况则大不相同。虽然一个最简单的方式是直接找到Dev-C++或者Visual Studio安装下来开箱即用,但是这样教无疑是极不负责的——上述两个程序虽然能够做到开箱即用,但是Dev-C++只能提供C/C++支持,可扩展性极差;Visual Studio更是称之为“.NET框架专属编译器”也不为过,它过于笨重,可扩展性也不强,而且它也只适合开发数万行甚至更多代码的大型项目。

因此,LCPU的共识是,推荐在Windows下使用Visual Studio Code作为代码编辑工具、GCC作为C++的编译器。这是因为VS Code是非常轻量级且插件非常丰富的文本编辑器,虽然不集成开发环境,但是其高自由度允许我们使用它进行几乎任何开发操作。

在Windows下安装这一工具并不容易。Windows有一个非常重要的系统变量“环境变量”,记录执行命令需要用到的可执行文件。而在手动安装程序的时候,往往不会自动把路径加到环境变量中。安装GCC就是一个典型例子。

在Windows上安装这些工具需要依次执行以下步骤:

\textbf{找到预编译版的GCC}。这个网上有很多相关的资源,在GitHub上搜索MinGW64也能够找到预编译版的GCC。下载并解压之,然后把它放进一个\textbf{完整路径没有中文的目录下}。这是因为GCC和中文路径是死敌,如果上述路径出现了中文大概率会出现问题。

\textbf{把GCC添加到环境变量}。一般地,这一步有两种操作方式:通过视窗操作和通过命令行操作。本人建议使用更加简明易懂的命令行进行相关操作。使用管理员权限打开一个新的终端,然后键入下列命令。该命令行认为你把MingGW64解压到了上述路径。如你放在其他路径,请自行更改路径。请务必使用半角符号,否则后果不堪设想!

\begin{codebox}[title=把GCC添加到环境变量]
    \$env:Path += ";C:\Program Files\mingw64\bin" # 实际操作请不要打最前面的反斜杠
\end{codebox}

为了检查安装是否成功,我们可以在命令行中输入如下命令:
\begin{codebox}
    gcc --version
\end{codebox}

如不报错,则说明安装成功了。

\textbf{安装VS Code}。这里推荐直接从官网安装之,且安装System版的VS Code。在安装该工具时,请务必选择“Add to PATH”选项。安装完成后,可以试一试开一个终端,输入“code”,如果能够打开VS Code,则说明安装成功,并成功添加到了环境变量。

\textbf{安装C/C++插件}。打开VS Code后,点击左侧的扩展按钮,搜索“C/C++”,然后安装紫色的、由Microsoft发布的三个即可。

\textbf{配置C/C++插件}。在VS Code中,按下Ctrl+Shift+P,输入“C/C++: Edit Configurations (UI)”,然后选择“C++ (GCC)”,然后在“Compiler path”中选择GCC的路径(例如C:\textbackslash Program Files\textbackslash mingw64\textbackslash bin\textbackslash g++.exe),然后点击“OK”即可。这样就能够正常使用C++编译器了。

\textbf{配置调试器(可选)}。在VS Code中,按下Ctrl+Shift+P,输入“C/C++: Edit Configurations (UI)”,然后选择“C++ (GDB)”,然后在“GDB path”中选择GDB的路径(例如C:\textbackslash Program Files\textbackslash mingw64\textbackslash bin\textbackslash gdb.exe),然后点击“OK”即可。这样就能够正常使用调试器进行视窗内逐行调试等功能了。

\subsubsection{MacOS}

\begin{codebox}[title=MacOS安装GCC编译器]   
    raise NotImplementedError("MacOS安装GCC编译器的过程")
\end{codebox}

\subsection{Python}

Python的安装则简单一些,但是深入使用较为困难。纵然问哦们可以直接安装Python并使用,但是通过此种形式安装的Python并不适合实际的开发工作:Python因为其生态的丰富性,调包是一个非常重要的操作。而对于不同的开发环境,我们可能需要不同的包或者相同包的不同版本,这往往会产生冲突。这时,虚拟环境就派上了用场。

以下内容均面向Windows系统;对于Mac将在新的一章中做介绍。

\subsubsection{虚拟环境及其安装}

虚拟环境指的是一个\textbf{隔离的}Python环境,其为每一个或者一组项目提供了一个专属空间。这样,对于刚刚提到的实际问题,我们就有了良好的解决方案:只需要给每一个项目创建一个虚拟环境,就可以避免冲突了。

目前一个最常用的虚拟环境管理器是Anaconda。其在Windows上的安装是类似于Code的;只需要在官网上找到Anaconda,并且确保将其添加到环境变量就可以了。你可以在终端中输入“conda”来进行检查。

为了创建并进入一个虚拟环境,你应该在终端中输入下列代码:

\begin{codebox}
    conda init powershell
    conda create -n MyEnv python==3.12
    conda activate MyEnv
\end{codebox}

上述命令中,conda init的意思是初始化适用于powershell的conda环境,该命令只需要执行一次,以后就再也不需要执行了。第二行中conda create的意思是创建一个虚拟环境,-n MyEnv的意思是该虚拟环境的名字是MyEnv,python==3.12的意思是在该虚拟环境下安装的Python解释器是3.12版本的。第三行的意思是激活该虚拟环境,此时在该终端下我们的操作就是在虚拟环境中进行了。

在虚拟环境中管理包,以前的一个方法是使用conda install package,不过这个方法现在已经不流行了。目前更加流行的办法是在conda内部使用pip进行该虚拟环境的包管理。

在上述条件下,我们创建的虚拟环境与包是安装在C盘的,这非常占用空间。关于怎样删除虚拟环境、怎样转移虚拟环境的位置,这留给同学们自行查找。

在VS Code上开发Python项目,我们依然需要安装一些插件。这个插件依然可以使用微软出品的一系列Python插件:Python Debugger、Python和Pylance。

要为运行脚本选择解释器的时候,我们要在右下角的按钮“选择解释器”中选择你使用的虚拟环境的解释器。不过一个更简单的方式是python ./*.py,VS Code会自动识别当前目录下的虚拟环境。

\chapter{文本处理基础} % Chapter Over
\begin{codebox}
    这段真就随便讲了
\end{codebox}


\section{Markdown}

\section{LaTeX}

\end{document}
